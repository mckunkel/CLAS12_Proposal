\documentclass[10pt,a4paper]{report}
\usepackage[latin1]{inputenc}
\usepackage{amsmath}
\usepackage{amsfonts}
\usepackage{amssymb}
\usepackage{graphicx}
\author{Michael C. Kunkel}
\begin{document}
	\section*{General}
	\begin{itemize}
		\item We have adjusted the rate to reflect a 80~kHz production instead of 140kHz.
		\item Statistical uncertainty has been added into the abstract, however we cannot precisely know the systematics yet. 
	\end{itemize}
	\section*{Aspects that need to be dealt with}
	\begin{itemize}
		\item Page 4: The double Dalitz decay does not contain more information than the single Dalitz decay. The measurement proposed for CLAS12 should be on the same weight as the proposed BESIII measurement, but the CLAS12 results is expected to be 1\% better in precision.
		\item Section 3: Added a footnote for the abbreviations. The text had already stated that a coincidence requirement.
		\item Page 17: Noted the change in wording
		\item Page 24: Reference [45] is a CLAS Masters Thesis, which is a valid reference. I have updated the bibliography to reflect. I will contact the appropriate CLAS member to get this thesis into the CLAS database.
		\item Section 4: Changed title to ''Proposed Measurement`` as requested.
		\item Page 25: The integrated photo-production cross section can be seen in Fig. 25 lower panel. Is the suggestion to place this figure elsewhere?
		\item Section 4: Rewrote sentence as prescribed.
		\item Trigger Requirements: Added text to reflect that the trigger particle to be $e^+$ or $e^-$ from Dalitz.
		\item **!!** We are unclear of your ratio 0.05/130. Figure 25 shows an average cross-section for photo-production to be $\sim$0.5$\mu b$. We believe that the confusion might be with the mis-labeling of Figure 25, where both figures are in W but one figure is mis-labeled in beam energy. We have fixed this discrepancy.
		\item Figure 25: Fixed x-axis labeling.
		\item Page 28: This method was only to approximate the production angles of $\eta'$. We assume that the low $Q^2$ production be that of real photo-production and scale accordingly. This will only underestimate the actual production scenario.
		\item We use G11 results from Mike Williams. We do not limit our calculation to 3.8GeV, we only present the available data that is to 3.8 GeV. Above 3.8 GeV we assume a $s^7$ scaling on the cross-section, but this only raises the total count rate by 5\% and thought it best to just present on the data available. The references for electro-production justify the $s^7$ scaling behavior. 
		\item Untagged Bremsstrahlung: Added a comment ''It should be noted that there will be $\eta'$ \ mesons produced from untagged Bremsstrahlung production which should be of the order $\approx 10$~kHz. This would increase the total Dalitz yield by $\approx$3000 events. ``
		\item Forward Tagger: Noted in section ''Detection of $e^+e^-$T Events`` that the Forward Tagger geometric acceptance was used in the FASTMC.
	\end{itemize}
\section*{Aspects that would improve the proposal}		
	\begin{itemize}
		\item Abstract: Revised as prescribed
		\item Section 2: Renamed to ''Kinematics of Decays`` Kept the ''Background from...`` in this section because it describes the kinematics of pair-production.
		\item Page 12: added uncertainty to abstract
		\item Section 3: Left section as is.
		\item CC and EC comparison: Changed to more descriptive title
		\item Plots 12-15: Left as is.
		\item Section 4: Left section as is
		\item Page 25: revised as prescribed
		\item Upper limit sentence: Removed as prescribed
		\item Production of $\eta'$...: Revised as prescribed.
		\item Page 31: revised as prescribed.
		\item Figure 29: Remade in ROOT
		\item Systematic uncertainties: We do not include low $M(ee)$ in the fit as we are not proposing to study the anomalous sector of meson. The relevant range is $M(ee)>M_{\pi^0}$. We have added a estimate of the systematic due to di-lepton acceptance by increasing the acceptance 5\% as a function of $M(ee)$
		
		
		
	\end{itemize}
\end{document}