\subsection{Dalitz Decay}\label{sec:dalitzdecay} 
When a pseudoscalar meson decays via a photon $\gamma$ and a dilepton ($l^{+}l^{-}$) pair, it is known as a Dalitz decay or a so-called single off-shell decay. The Dalitz decay is related to the two photon decay. However, in the Dalitz decay, one of the photons is off-shell ($\gamma^*$) and decays into a dilepton pair. Since the Dalitz decay is related to the two photon decay, the form factor of the Dalitz decay, for P($\eta'$), will be similar to the form factor of the two photon decay of P( $\eta'$), except there will be an effective mass dependence for the Dalitz decay. Figure~\ref{fig:piz.dalitz} depicts the Feymann diagram of the Dalitz decay.

The amplitude for the decay $P_P \to \gamma^\star(p) \gamma(k) \to l^+(p_+)l^-(p_-) \gamma(k)$ is given by the following expression:
\begin{equation}\label{eq:piz.eeg.amp}
{\cal M}(P\to l^+(p_+,s_+)l^-(p_-,s_-) \gamma) = {M}_P(p^2,k^2=0) \varepsilon_{\mu\nu\rho\sigma} \frac{1}{q^2} e \bar u(p_-,s_-) \gamma^\mu v(p_+,s_-) q^\nu \epsilon^\rho k^\sigma.
\end{equation}
Comparing the amplitudes of Eq.~\ref{eq:piz.eeg.amp} and Eq.~\ref{eq:piz.gg.amp} it is seen that the polarization of the off-shell photon turned into the current $e \bar u(p_-,s_-) \gamma^\mu v(p_+,s_-)$ of the lepton pair. The parameters $s_\pm$ are the spin helicities of the outgoing leptons $l^\pm$ and as in  Eq.~\ref{eq:piz.gg}, $\epsilon$ is the polarization of the outgoing photon. 
%
\subsubsection{\emph{Squared Matrix Element}}


\begin{align}\label{eq:piz.eeg}
& \left|{\cal M}(P\to l^+(p_+,s_+)l^-(p_-,s_-) \gamma)\right|^2 = \nonumber \\ & \frac{e^2}{q^4} \left|M\right|^2  \varepsilon_{\mu\nu\rho\sigma}\varepsilon_{\mu^{\prime}\nu^{\prime}\rho^{\prime}\sigma^{\prime}}\bar u(p_-,s_-) \gamma^\mu v(p_+,s_+) \bar v(p_+,s_+) \gamma^{\mu^{\prime}}  u(p_-,s_-) q^\nu \epsilon^\rho k^\sigma q^{\nu^{\prime}} \epsilon^{\rho^{\prime}} k^{\sigma^{\prime}} .
%
\end{align}
using an equation found between equation 5.3 and 5.4 found in~\cite{peskin}
\begin{align}\label{eq:spin.sum}
& \sum\limits_{s_{-},s_{+}}^{} \bar{u}(p_{-},s_{-})\gamma^{\mu}\nu(p_{+},s_{+})\bar{\nu}(p_{+},s_{+})\gamma^{\mu^{\prime}}u(p_{-},s_{-}) = Tr\left[ (\slashed{p}_- +m)\gamma^{\mu} (\slashed{p}_+-m)\gamma^{\mu^{\prime}} \right]\nonumber \\ & =2q^{2}\left[-(g_{\mu\mu^{\prime}}-\frac{p_{\mu}p_{\mu^{\prime}}}{q^{2}} ) - \frac{(p_{+} - p_{-})_{\mu}(p_{+} - p_{-})_{\mu^{\prime}}}{q^{2}}\right]
\end{align}
where the identity $q = p_+ + p_-$ was used.
Substituting Eq.~\ref{eq:spin.sum} into Eq.~\ref{eq:piz.eeg}
\begin{align} \label{eq:piz.eeg.midway1}
\left|{\cal M}\right|^{2} =\frac{2e^{2}\left|M_{P}\right|^{2}}{q^{2}}\varepsilon_{\mu\nu\rho\sigma}\varepsilon_{\mu^{\prime}\nu^{\prime}\rho^{\prime}\sigma^{\prime}}\left[-g^{\mu\mu^{\prime}} - \frac{(p_{+} - p_{-})^{\mu}(p_{+} - p_{-})^{\mu^{\prime}}}{q^{2}}\right](-g^{\nu\nu^{\prime}})q^{\rho}k^{\sigma}q^{\rho^{\prime}}k^{\sigma^{\prime}}
\end{align}
Substituting $k = P - q$ and $p_- = q - p_+$ into Eq.~\ref{eq:piz.eeg.midway1}
\begin{align} \label{eq:piz.eeg.midway2}
\left|{\cal M}\right|^{2} = & \frac{2e^{2}\left|M_{P}\right|^{2}}{q^{2}}\varepsilon_{\mu\nu\rho\sigma}\varepsilon_{\mu^{\prime}\nu^{\prime}\rho^{\prime}\sigma^{\prime}}\left[-g^{\mu\mu^{\prime}} - \frac{(2p_{+} - q)^{\mu}(2p_{+} - q)^{\mu^{\prime}}}{q^{2}}\right] \nonumber \\ & \times (-g^{\nu\nu^{\prime}})      
(q^{\rho}P^{\sigma} - q^{\rho}q^{\sigma}) (q^{\rho}P^{\sigma^{\prime}} - q^{\rho^{\prime}}q^{\sigma^{\prime}})
\end{align}
Applying properties of $-g^{\mu\mu^{\prime}}$ and $-g^{\nu\nu^{\prime}}$ onto Eq.~\ref{eq:piz.eeg.midway2}
\begin{align} \label{eq:piz.eeg.midway3}
\left|{\cal M}\right|^{2} = & \frac{2e^{2}\left|M_{P}\right|^{2}}{q^{2}}
\left[\varepsilon_{\mu\nu\rho\sigma}\varepsilon^{\mu\nu}_{\quad \rho^{\prime}\sigma^{\prime}}q^{\rho}P^{\sigma}q^{\rho^{\prime}}P^{\sigma^{\prime}} + \frac{4}{q^2} \varepsilon_{\mu\nu\rho\sigma}\varepsilon^{\mu}_{\ \ \nu^{\prime} \rho^{\prime}\sigma^{\prime}} p_{+}^{\nu}p_{+}^{\nu^{\prime}}q^{\rho}q^{\rho^{\prime}}P^{\sigma}P^{\sigma^{\prime}}\right]
\end{align}
Switching to the rest frame of the pseudoscalar meson, $P_p$, the 4-momenta is transformed to $P^\sigma = m_p\delta^{\sigma 0}$. The squared amplitude of Eq.~\ref{eq:piz.eeg.midway3} reads;
\begin{align} \label{eq:piz.eeg.midway4}
\left|{\cal M}\right|^{2} = & \frac{2e^{2}\left|M_{P}\right|^{2}}{q^{2}}m_p^2
\left[\varepsilon_{\mu\nu\rho}\varepsilon^{\mu\nu}_{\ \ \rho^{\prime}}q^{\rho}q^{\rho^{\prime}} - \frac{4}{q^2} \varepsilon_{\mu\nu\rho}\varepsilon^{\mu}_{\ \nu^{\prime}\rho^{\prime}} p_{+}^{\nu}p_{+}^{\nu^{\prime}}q^{\rho}q^{\rho^{\prime}}\right]
\end{align}
The sign change is due to $g^{\sigma \sigma^{\prime}} = -\delta^{\sigma \sigma^{\prime}}$. 
Using the antisymmetric tensor properties $\varepsilon_{\mu\nu\rho}\varepsilon^{\mu\nu}_{\ \ \rho^{\prime}} = 2\delta_{\rho\rho^{\prime}}$ and $\varepsilon_{\mu\nu\rho}\varepsilon^{\mu}_{\ \nu^{\prime}\rho^{\prime}} = \delta_{\nu\nu^{\prime}}\delta_{\rho\rho^{\prime}} - \delta_{\nu\rho^{\prime}}\delta_{\rho\nu^{\prime}} = (\hat{e}_{\nu} \times \hat{e}_{\rho}) \cdot (\hat{e}_{\nu^{\prime}} \times \hat{e}_{\rho^{\prime}})$, Eq.~\ref{eq:piz.eeg.midway4} is reduced to 
\begin{align} \label{eq:piz.eeg.final}
\left|{\cal M}\right|^{2} =  \frac{2e^{2}\left|M_{P}\right|^{2}}{q^{2}}m_p^2
\left[2\left|\bf{q}\right|^2 - \frac{4}{q^2} \left|\bf{q}\right|^2 \left|\bf{p_{+}}\right|^2 \sin^2(\theta_{p_{_+}q}) \right]
\end{align}

\subsubsection{\emph{Decay rate}}
The decay rate of a three-body decay is given in Equation 46.19 of~\cite{pdg2014} as
\begin{align}\label{eq:pdg.3body}
d\Gamma = \frac{1}{(2 \pi)^5} \frac{1}{16 m_p^2} \left|{\cal M}\right|^2 \left|\bf{p_1^*}\right| \left|\bf{p_3}\right|d\Omega_1^*d\Omega_3 dm_{12} \ ,
\end{align}
%
where ($\left|\bf{p_1^*}\right|,\Omega_1^*$) is the momentum of particle 1 in the rest frame of 1 and 2, and $\Omega_3$ is the angle of particle 3 in the rest frame of the decaying particle $m_p$~\cite{pdg2014}. Relating Eq.~\ref{eq:pdg.3body} to the variables in Eq.~\ref{eq:piz.eeg.final}, where $(\left|\bf{p_1^*}\right|,\Omega_1^*) = (\left|\bf{p_+}\right|,\Omega_{p_{_+}q})$, $m_{12} = q$ and $(\left|\bf{p_3}\right|,\Omega_3) = (\left|\bf{p_k}\right|,\Omega_k)$, reads;
\begin{align}\label{eq:pdg.3body.sub}
d\Gamma = \frac{1}{(2 \pi)^5} \frac{1}{16 m_p^2} \left|{\cal M}\right|^2 \left|\bf{p_+}\right| \left|\bf{p_k}\right|d\Omega_+d\Omega_k dq \ ,
\end{align}
%
In the rest from of the decaying particle $m_p$, the 3-momenta $\left|\bf{p_k}\right| = \left|\bf{q}\right|$ and the solid angle $\Omega_k = \Omega_q$. Substituting the square matrix element from Eq.~\ref{eq:piz.eeg.final} into Eq.~\ref{eq:pdg.3body.sub} yields;
%
\begin{align}\label{eq:pdg.3body.sub2}
d\Gamma = \frac{1}{(2 \pi)^5} \frac{1}{16 m_p^2} \frac{2e^{2}\left|M_{P}\right|^{2}}{q^{2}}m_p^2
\left[2\left|\bf{q}\right|^2 - \frac{4}{q^2} \left|\bf{q}\right|^2 \left|\bf{p_{+}}\right|^2 \sin^2(\theta_{p_{_+}q}) \right] \left|\bf{p_+}\right| \left|\bf{q}\right|d\Omega_{p_{_+}q}d\Omega_q dq\ .
\end{align}
The variables $\left|\bf{q}\right|$ and $\left|\bf{p_+}\right|$ can be redefined, by means of Eq.~46.20b and Eq.~46.20a of~\cite{pdg2014}, as 
\begin{align}
\left|\bf{q}\right| = \frac{m_p^2 - q^2}{2m_p} \label{eq:eeg.qeq} \\
\left|\bf{p_+}\right| = \frac{\sqrt{q^2 - 4m_l^2}}{2} = \frac{q\sqrt{1 - \frac{4m_l^2}{q^2} } } {2} =\frac{q {\cal K}  } {2}  \label{eq:eeg.p+eq} \ ,
\end{align} 
where ${\cal K} = \sqrt{1 - \frac{4m_l^2}{q^2}}$. Replacing the variables calculated in Eq.~\ref{eq:eeg.qeq} and Eq.~\ref{eq:eeg.p+eq} into Eq.~\ref{eq:pdg.3body.sub2} and collecting terms yields;
\begin{align}\label{eq:pdg.3body.sub3}
d\Gamma = \frac{1}{(2 \pi)^5} \frac{1}{16 m_p^2} \left|M_{P}\right|^{2} \left[ \frac{2e^2 m_p^2}{8} \left( \frac{m_p^2 - q^2}{2 m_p}\right)^3\right]\left( 2 -{\cal K}^2\sin^2(\theta_{p_{_+}q})\right)\frac{{\cal K}}{4 q^2}dq^2d\Omega_{p_{_+}q}d\Omega_q \ ,
\end{align}
where the identity $qdq = \frac{dq^2}{2}$. Performing the integration of $\Omega_{p_{_+}q}d\Omega_q$ and replacing $e^2 = 4\pi\alpha$ transforms Eq.~\ref{eq:pdg.3body.sub3} into;
\begin{align}\label{eq:pdg.3body.sub4}
d\Gamma = \frac{1}{(2 \pi)^3} \frac{1}{32} \frac{4 \pi \alpha}{3} \left|M_{P}\right|^{2} \left[ \frac{m_p^6 \left( 1- \frac{q^2}{m_p^2}\right)^3}{m_p^3} \right]\left( 3 -{\cal K}^2\right)\frac{{\cal K}}{q^2}dq^2\ ,
\end{align}
which can be simplified further to;
\begin{align}\label{eq:eeg.final}
d\Gamma = \left(\frac{1}{64\pi} \left|M_{P}\right|^{2}m_{P}^{3} \right) \frac{2 \alpha}{3 \pi} \frac{1}{q^2} \left( 1- \frac{q^2}{m_p^2}\right)^3 \left( 1+ \frac{2m_l^2}{q^2}\right) \left( 1- \frac{4m_l^2}{q^2}\right)^{\frac{1}{2}} dq^2\ .
\end{align}
%
The form factor ${M}_P(p^2,k^2=0)$ can be written as follows:
\begin{align}
 {M}_P \to {M}_P \times \left|F(q^2)\right| \ ,
\end{align}
where $M_p$ is the decay constant of two photons mentioned in Sec.~\ref{sec:piz.gg} and $\left|F(q^2)\right|$ is called the transition form factor, which defines the electromagnetic space structure of the meson. 

It can be seen that the first set of variables in parenthesis in Eq.~\ref{eq:eeg.final} is Eq.~\ref{eq:piz.gg.decay.final}, therefore;
\begin{align}\label{eq:eegff.final}
\frac{d\Gamma}{\Gamma_{\gamma\gamma} dq^2} = \frac{2 \alpha}{3 \pi} \frac{1}{q^2} \left( 1- \frac{q^2}{m_p^2}\right)^3 \left( 1+ \frac{2m_l^2}{q^2}\right) \left( 1- \frac{4m_l^2}{q^2}\right)^{\frac{1}{2}} \left|F(q^2)\right|^2 \ ,
\end{align}
which is the Kroll-Wada equation founded in~\cite{KrollWada}.

The value of $\left|F(q^2)\right|$ can be directly measured by comparison of the differential cross section with that of Q.E.D. pointlike differential cross section i.e.
\begin{align}
\frac{d\sigma}{dq^{2}} = \left[\frac{d\sigma}{dq^{2}}\right]_{\text{pointlike}}\left| F(q^{2})\right| ^{2}\nonumber \ ,
\end{align}
or by performing a line shape analysis on the $l^{+}l^{-}$ invariant system using assumptions on the structure of $\left|F(q^2)\right|$. One such assumption for $\left|F(q^2)\right|$ is the dipole approximation in which 
\begin{align}
F(q^{2}) = \frac{\Lambda^2(\Lambda^2 + \gamma^2)}{(\Lambda^{2} - q^2) + \Lambda^2\gamma^2 } \nonumber
\end{align}
where the parameters $\Lambda$ and $\gamma$ correspond to the mass and width of the Breit-Wigner shape for the effective contributing vector meson. A first approximation is that $\Lambda \approx M_{\rho} \approx 0.7$~GeV and $\gamma \approx \Gamma_{\rho}  \approx 0.12$~GeV
\subsection{Summary}
The two photon decay and the Dalitz decay have different branching ratios. This difference is attributed to the factor of $\alpha$ along with a $q^2$ dependence calculated in the Dalitz decay. However, due to the probability of a photon converting into an electron-positron pair in $\ell$H$_2$, the total amount of \epem pairs produced via photon conversion can contaminate the measurement of the form factor if primary vertex constraints are not used.
 